\section{مقاومت در برابر دیس اسمبلی با استفاده از تکنیک‌های رمزگذاری}




با تبدیل فایل اجرایی به کد اسمبلی، تحلیل‌گران می‌توانند منطق داخلی، باگ‌ها و بدافزارهای احتمالی را شناسایی کنند\cite{ref1}. به‌عبارت دیگر، دیس‌اسمبلی امکان افشای عملکرد برنامه و پیدا کردن آسیب‌پذیری‌ها یا کدهای مخرب را فراهم می‌کند. به همین دلیل، نویسندگان نرم‌افزارهای محافظت‌شده تلاش می‌کنند با روش‌هایی مانند رمزگذاری و ابهام‌زایی، دیس‌اسمبلی را دشوار کنند.


\subsection{کد پلی‌مورفیک (\lr{Polymorphic Code})}

\noindent \textbf{تعریف و مکانیسم: }

کد پلی‌مورفیک نوعی کد خودتغییرده است که در هر نسخه ظاهر باینری متفاوتی دارد، ولی منطق و الگوریتم اصلی‌اش حفظ می‌شود\cite{ref3}. معمولاً در این روش، بخش اصلی کد («\lr{payload}») با یک کلید تصادفی رمزگذاری می‌شود و یک بخش کوتاه کوچک در ابتدای برنامه (بخش رمزگشا) وجود دارد که با کلید مربوطه، payload را در حافظه رمزگشایی می‌کند\cite{ref3, ref4}. به‌عنوان مثال، ویروس‌های پلی‌مورفیک مشهور، از رمزنگاری XOR با یک کلید تصادفی برای مخفی کردن بدنه‌ی اصلی خود استفاده می‌کنند، به‌طوری که هر نمونه از لحاظ باینری کاملاً با نمونه‌های دیگر متفاوت است\cite{ref4}.

\noindent \textbf{مزایا و کاربردها: }

مزیت اصلی پلی‌مورفیسم مقاومت در برابر شناسایی مبتنی بر امضاست؛ چون هر بار که کد اجرا می‌شود، الگویی یکتا ایجاد می‌کند و امضای ثابتی ندارد\cite{ref4}. این روش عمدتاً در ویروس‌ها، کرم‌ها و بدافزارها به کار می‌رود تا برنامه‌های ضدویروس را فریب دهد. همچنین در برخی محافظت‌کننده‌های تجاری کد (پکرها) دیده می‌شود که با تولید کلیدهای مختلف یا تغییر الگوریتم رمزگشایی، کد را در هر توزیع متفاوت می‌کنند. پلی‌مورفیسم نسبتاً ساده است و تأثیر زیادی بر عملکرد اجرایی برنامه ندارد.

\noindent \textbf{نحوه عملکرد: }

معمولاً هر نمونه پلی‌مورفیک شامل دو بخش است: یک لایه رمزنگاری (برای پنهان‌سازی \lr{payload}) و یک بخش رمزگشای کوتاه. در زمان اجرا، ابتدا بخش رمزگشا فعال شده و بدنه اصلی را در حافظه بازمی‌کند، سپس کنترل به کد اصلی داده می‌شود\cite{ref3}. به خاطر این ساختار، دیس‌اسمبلی ایستا مؤثر نیست زیرا کد اصلی در فایل رمزیافته و فقط با اجرای برنامه آشکار می‌شود\cite{ref3}.

\noindent \textbf{معایب و روش‌های تحلیل: }

گرچه پلی‌مورفیسم شناسایی ساده مبتنی بر امضا را دشوار می‌کند، اما شکستن آن آسان است. تحلیل‌گران می‌توانند برنامه را اجرا کنند و با قرار دادن نقطه‌نگاری روی بخش رمزگشا یا استفاده از دیباگر، بدنه اصلی را در زمان اجرای برنامه مشاهده و استخراج نمایند\cite{ref3}. افزون بر این، چون الگوریتم رمزگشایی معمولاً ثابت یا ساده است، می‌توان با نوشتن اسکریپت‌های کوچک یا استفاده از ابزارهای خودکار، کد را رمزگشایی کرد. به همین علت، اکثر روش‌های مقابله با کد پلی‌مورفیک بر تحلیل پویا (اجرا در محیط کنترل‌شده و خاموش کردن رمزنگاری) و یا نرمال‌سازی کد بر مبنای الگوریتم‌های آماری متمرکزند.


\subsection{کد متامورفیک (\lr{Metamorphic Code})}

\noindent \textbf{تعریف و مکانیسم: }

کد پلی‌مورفیک نوعی کد خودتغییرده است که در هر نسخه ظاهر باینری متفاوتی دارد، ولی منطق و الگوریتم اصلی‌اش حفظ می‌شود\cite{ref3}. معمولاً در این روش، بخش اصلی کد («\lr{payload}») با یک کلید تصادفی رمزگذاری می‌شود و یک بخش کوتاه کوچک در ابتدای برنامه (بخش رمزگشا) وجود دارد که با کلید مربوطه، payload را در حافظه رمزگشایی می‌کند\cite{ref4}\cite{ref3}. به‌عنوان مثال، ویروس‌های پلی‌مورفیک مشهور، از رمزنگاری XOR با یک کلید تصادفی برای مخفی کردن بدنه‌ی اصلی خود استفاده می‌کنند، به‌طوری که هر نمونه از لحاظ باینری کاملاً با نمونه‌های دیگر متفاوت است\cite{ref4}.

\noindent \textbf{مزایا و کاربردها: }

مزیت اصلی پلی‌مورفیسم مقاومت در برابر شناسایی مبتنی بر امضاست؛ چون هر بار که کد اجرا می‌شود، الگویی یکتا ایجاد می‌کند و امضای ثابتی ندارد\cite{ref4}. این روش عمدتاً در ویروس‌ها، کرم‌ها و بدافزارها به کار می‌رود تا برنامه‌های ضدویروس را فریب دهد. همچنین در برخی محافظت‌کننده‌های تجاری کد (پکرها) دیده می‌شود که با تولید کلیدهای مختلف یا تغییر الگوریتم رمزگشایی، کد را در هر توزیع متفاوت می‌کنند. پلی‌مورفیسم نسبتاً ساده است و تأثیر زیادی بر عملکرد اجرایی برنامه ندارد.

\noindent \textbf{معایب و روش‌های تحلیل: }

گرچه پلی‌مورفیسم شناسایی ساده مبتنی بر امضا را دشوار می‌کند، اما شکستن آن آسان است. تحلیل‌گران می‌توانند برنامه را اجرا کنند و با قرار دادن نقطه‌نگاری روی بخش رمزگشا یا استفاده از دیباگر، بدنه اصلی را در زمان اجرای برنامه مشاهده و استخراج نمایند\cite{ref3}. افزون بر این، چون الگوریتم رمزگشایی معمولاً ثابت یا ساده است، می‌توان با نوشتن اسکریپت‌های کوچک یا استفاده از ابزارهای خودکار، کد را رمزگشایی کرد. به همین علت، اکثر روش‌های مقابله با کد پلی‌مورفیک بر تحلیل پویا (اجرا در محیط کنترل‌شده و خاموش کردن رمزنگاری) و یا نرمال‌سازی کد بر مبنای الگوریتم‌های آماری متمرکزند.



\subsection{بسته‌سازی کد (\lr{Packers})}

\noindent \textbf{تعریف و مکانیسم: }

کد پلی‌مورفیک نوعی کد خودتغییرده است که در هر نسخه ظاهر باینری متفاوتی دارد، ولی منطق و الگوریتم اصلی‌اش حفظ می‌شود\cite{ref3}. معمولاً در این روش، بخش اصلی کد («payload») با یک کلید تصادفی رمزگذاری می‌شود و یک بخش کوتاه کوچک در ابتدای برنامه (بخش رمزگشا) وجود دارد که با کلید مربوطه، payload را در حافظه رمزگشایی می‌کند\cite{ref4}\cite{ref3}. به‌عنوان مثال، ویروس‌های پلی‌مورفیک مشهور، از رمزنگاری XOR با یک کلید تصادفی برای مخفی کردن بدنه‌ی اصلی خود استفاده می‌کنند، به‌طوری که هر نمونه از لحاظ باینری کاملاً با نمونه‌های دیگر متفاوت است\cite{ref4}.

\noindent \textbf{مزایا و کاربردها: }

مزیت اصلی پلی‌مورفیسم مقاومت در برابر شناسایی مبتنی بر امضاست؛ چون هر بار که کد اجرا می‌شود، الگویی یکتا ایجاد می‌کند و امضای ثابتی ندارد\cite{ref4}. این روش عمدتاً در ویروس‌ها، کرم‌ها و بدافزارها به کار می‌رود تا برنامه‌های ضدویروس را فریب دهد. همچنین در برخی محافظت‌کننده‌های تجاری کد (پکرها) دیده می‌شود که با تولید کلیدهای مختلف یا تغییر الگوریتم رمزگشایی، کد را در هر توزیع متفاوت می‌کنند. پلی‌مورفیسم نسبتاً ساده است و تأثیر زیادی بر عملکرد اجرایی برنامه ندارد.

\noindent \textbf{معایب و روش‌های تحلیل: }

گرچه پلی‌مورفیسم شناسایی ساده مبتنی بر امضا را دشوار می‌کند، اما شکستن آن آسان است. تحلیل‌گران می‌توانند برنامه را اجرا کنند و با قرار دادن نقطه‌نگاری روی بخش رمزگشا یا استفاده از دیباگر، بدنه اصلی را در زمان اجرای برنامه مشاهده و استخراج نمایند\cite{ref3}. افزون بر این، چون الگوریتم رمزگشایی معمولاً ثابت یا ساده است، می‌توان با نوشتن اسکریپت‌های کوچک یا استفاده از ابزارهای خودکار، کد را رمزگشایی کرد. به همین علت، اکثر روش‌های مقابله با کد پلی‌مورفیک بر تحلیل پویا (اجرا در محیط کنترل‌شده و خاموش کردن رمزنگاری) و یا نرمال‌سازی کد بر مبنای الگوریتم‌های آماری متمرکزند.




\subsection{ابهام‌زایی مبتنی بر مجازی‌سازی (\lr{Virtualization-Based Obfuscation})}

\noindent \textbf{تعریف و مکانیسم: }

کد پلی‌مورفیک نوعی کد خودتغییرده است که در هر نسخه ظاهر باینری متفاوتی دارد، ولی منطق و الگوریتم اصلی‌اش حفظ می‌شود\cite{ref3}. معمولاً در این روش، بخش اصلی کد («payload») با یک کلید تصادفی رمزگذاری می‌شود و یک بخش کوتاه کوچک در ابتدای برنامه (بخش رمزگشا) وجود دارد که با کلید مربوطه، payload را در حافظه رمزگشایی می‌کند\cite{ref4}\cite{ref3}. به‌عنوان مثال، ویروس‌های پلی‌مورفیک مشهور، از رمزنگاری XOR با یک کلید تصادفی برای مخفی کردن بدنه‌ی اصلی خود استفاده می‌کنند، به‌طوری که هر نمونه از لحاظ باینری کاملاً با نمونه‌های دیگر متفاوت است\cite{ref4}.

\noindent \textbf{مزایا و کاربردها: }

مزیت اصلی پلی‌مورفیسم مقاومت در برابر شناسایی مبتنی بر امضاست؛ چون هر بار که کد اجرا می‌شود، الگویی یکتا ایجاد می‌کند و امضای ثابتی ندارد\cite{ref4}. این روش عمدتاً در ویروس‌ها، کرم‌ها و بدافزارها به کار می‌رود تا برنامه‌های ضدویروس را فریب دهد. همچنین در برخی محافظت‌کننده‌های تجاری کد (پکرها) دیده می‌شود که با تولید کلیدهای مختلف یا تغییر الگوریتم رمزگشایی، کد را در هر توزیع متفاوت می‌کنند. پلی‌مورفیسم نسبتاً ساده است و تأثیر زیادی بر عملکرد اجرایی برنامه ندارد.

\noindent \textbf{معایب و روش‌های تحلیل: }

گرچه پلی‌مورفیسم شناسایی ساده مبتنی بر امضا را دشوار می‌کند، اما شکستن آن آسان است. تحلیل‌گران می‌توانند برنامه را اجرا کنند و با قرار دادن نقطه‌نگاری روی بخش رمزگشا یا استفاده از دیباگر، بدنه اصلی را در زمان اجرای برنامه مشاهده و استخراج نمایند\cite{ref3}. افزون بر این، چون الگوریتم رمزگشایی معمولاً ثابت یا ساده است، می‌توان با نوشتن اسکریپت‌های کوچک یا استفاده از ابزارهای خودکار، کد را رمزگشایی کرد. به همین علت، اکثر روش‌های مقابله با کد پلی‌مورفیک بر تحلیل پویا (اجرا در محیط کنترل‌شده و خاموش کردن رمزنگاری) و یا نرمال‌سازی کد بر مبنای الگوریتم‌های آماری متمرکزند.



\subsection{رمزگذاری کد در زمان اجرا (\lr{Runtime Code Encryption})}

\noindent \textbf{تعریف و مکانیسم: }

کد پلی‌مورفیک نوعی کد خودتغییرده است که در هر نسخه ظاهر باینری متفاوتی دارد، ولی منطق و الگوریتم اصلی‌اش حفظ می‌شود\cite{ref3}. معمولاً در این روش، بخش اصلی کد («payload») با یک کلید تصادفی رمزگذاری می‌شود و یک بخش کوتاه کوچک در ابتدای برنامه (بخش رمزگشا) وجود دارد که با کلید مربوطه، payload را در حافظه رمزگشایی می‌کند\cite{ref3, ref4}. به‌عنوان مثال، ویروس‌های پلی‌مورفیک مشهور، از رمزنگاری XOR با یک کلید تصادفی برای مخفی کردن بدنه‌ی اصلی خود استفاده می‌کنند، به‌طوری که هر نمونه از لحاظ باینری کاملاً با نمونه‌های دیگر متفاوت است\cite{ref4}.

\noindent \textbf{مزایا و کاربردها: }

مزیت اصلی پلی‌مورفیسم مقاومت در برابر شناسایی مبتنی بر امضاست؛ چون هر بار که کد اجرا می‌شود، الگویی یکتا ایجاد می‌کند و امضای ثابتی ندارد\cite{ref4}. این روش عمدتاً در ویروس‌ها، کرم‌ها و بدافزارها به کار می‌رود تا برنامه‌های ضدویروس را فریب دهد. همچنین در برخی محافظت‌کننده‌های تجاری کد (پکرها) دیده می‌شود که با تولید کلیدهای مختلف یا تغییر الگوریتم رمزگشایی، کد را در هر توزیع متفاوت می‌کنند. پلی‌مورفیسم نسبتاً ساده است و تأثیر زیادی بر عملکرد اجرایی برنامه ندارد.

\noindent \textbf{معایب و روش‌های تحلیل: }

گرچه پلی‌مورفیسم شناسایی ساده مبتنی بر امضا را دشوار می‌کند، اما شکستن آن آسان است. تحلیل‌گران می‌توانند برنامه را اجرا کنند و با قرار دادن نقطه‌نگاری روی بخش رمزگشا یا استفاده از دیباگر، بدنه اصلی را در زمان اجرای برنامه مشاهده و استخراج نمایند\cite{ref3}. افزون بر این، چون الگوریتم رمزگشایی معمولاً ثابت یا ساده است، می‌توان با نوشتن اسکریپت‌های کوچک یا استفاده از ابزارهای خودکار، کد را رمزگشایی کرد. به همین علت، اکثر روش‌های مقابله با کد پلی‌مورفیک بر تحلیل پویا (اجرا در محیط کنترل‌شده و خاموش کردن رمزنگاری) و یا نرمال‌سازی کد بر مبنای الگوریتم‌های آماری متمرکزند.